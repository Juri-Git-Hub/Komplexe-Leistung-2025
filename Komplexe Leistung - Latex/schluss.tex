\section{Schluss}

In der vorliegenden Komplexen Leistung wurde die Lösung der Krocket-Aufgabe detailliert analysiert und optimiert. Die zentrale Herausforderung bestand darin, einen effizienten, linearen Algorithmus zu entwickeln, der überprüft, ob ein Ball in der vorgegebenen Reihenfolge mit einem einzigen Schuss durch alle Tore geschossen werden kann. Durch den Einsatz von Konzepten wie Winkeltests, Isoklinen und verzerrten Ebenen konnte die Laufzeitkomplexität erheblich verbessert werden. Der entwickelte Ansatz stellt somit eine bedeutende Verbesserung gegenüber dem ursprünglichen Brute-Force-Algorithmus dar.

Die Zielsetzung der Arbeit wurde erreicht: Es wurde ein Algorithmus entwickelt, der es ermöglicht, für beliebige Konfigurationen von Toren eine Lösung effizient zu berechnen oder nachzuweisen, dass keine Lösung existiert.

Mir hat die Erarbeitung der Komplexen Leistung gezeigt, wie bedeutend eine systematische Herangehensweise an komplexe Fragestellungen ist. Durch die Fehleranalyse und kontinuierliche Optimierung wurde mir immer wieder verdeutlicht, wie essenziell eine präzise und strukturierte Denkweise in der Informatik ist. Die eigenständige Einarbeitung hat meine persönliche Kompetenz in diesen Bereichen erheblich gesteigert.

Die Erkenntnisse dieser Arbeit besitzen nicht nur theoretische Relevanz, sondern lassen sich auch in praktischen Anwendungen der Robotik nutzen. Ein konkretes Beispiel hierfür ist die Laufpfaderkennung autonomer Roboter, die – ähnlich wie in der Krocket-Aufgabe – Hindernisse in einer bestimmten Reihenfolge passieren müssen. Die Konzepte der Chokepoints und Isoklinen lassen sich auf die Bewegungsplanung und Navigation dieser Roboter übertragen. Das Team der HTWK-Robots, dem ich angehöre, prüft derzeit mögliche Integrationen des Algorithmus in der Laufpfaderkennung sowie in der Schussplanung, da die Roboter zunächst feststellen müssen, ob ein Schusspfad nicht durch Hindernisse blockiert ist.

Eine mögliche Erweiterung des Algorithmus, die die Option eines doppelten Schusses berücksichtigt, wäre sinnvoll. So wäre die Planung von Passspielen oder komplexeren Manövern denkbar. Dies würde neue Herausforderungen hinsichtlich der optimalen Schussaufteilung und strategischen Positionierung mit sich bringen. Auch eine Berechnung der Schusskraft wäre hilfreich, um zu bestimmen, ob der Schuss physisch überhaupt realisierbar ist.

\textbf{Zusammenfassend} bietet diese Arbeit eine algorithmische Lösung für eine anspruchsvolle Problemstellung. Der entwickelte Algorithmus besitzt vielfältige Anwendungsmöglichkeiten, beispielsweise in der Laufpfadberechnung, und bietet Potenzial für zukünftige Weiterentwicklungen. Die vorliegende Komplexe Leistung zeigt darüber hinaus, wie augenscheinlich einfache Probleme keineswegs einfache Lösungen haben müssen, da diese entweder sehr rechenintensiv wie der Brute Force Algorithmus sind oder erheblicheren konzeptuellen Aufwand erfordern.