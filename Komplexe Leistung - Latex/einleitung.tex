\section{Einleitung}

Als ich in der 3. Klasse war, haben mir meine Eltern einen \enquote{Calliope mini} geschenkt, einen einfachen, mit einer Blocksprache ähnlich zu Scratch, selbst programmierbaren Minicomputer. Diese frühe Einführung in die Welt der Informatik weckte eine Begeisterung, die mich bis heute nicht loslässt und mich letztes Jahr zum Hochschulteam HTWK Robots führte. Dort programmieren wir humanoide Roboter zum Fußballspielen.

Mit Lennart Peters, ebenfalls Schüler und Teammitglied der HTWK Robots, nahm ich Ende 2024 am 43. Bundeswettbewerb Informatik, zusammen mit insgesamt 1790 Jugendlichen teil \cite{teilnehmerzahl}. Max Polter, ebenfalls ein Teammitglied der HTWK Robots war unser Betreuer. Die erste Runde des Wettbewerbs umfasste 5 Aufgaben, von denen mindestens 3 bearbeitet werden mussten. Teilnehmer bis zur 10. Klasse durften dabei auch zwischen zwei Junioraufgaben auswählen \cite{aufgaben}. Für mein Teammitglied Lennart Peters und mich war es die erste Teilnahme an diesem Wettbewerb. Wir waren beide begeistert, unsere Programmier- und Informatikkenntnisse in einem Wettbewerb zeigen zu dürfen.

Besonders faszinierte uns die 4. Aufgabe \enquote{Krocket}, welche die anspruchsvollste Aufgabe mit einer durchschnittlichen Bewertung von nur 2,8 von 5 Punkten war, im Vergleich dazu wurde die Junioraufgabe 2 durchschnittlich mit 4,2 von 5 Punkten bewertet und insgesamt von 200 Jugendlichen mehr bearbeitet und eingereicht \cite{anschreiben}. In der Aufgabe sollte überprüft werden, ob es möglich ist, mit einem einzigen geradlinigen Schlag einen Ball mit gegebenem Ballradius so zu schießen, dass er in der vorgegebenen Reihenfolge alle Tore passiert. Auch wenn sich die Aufgabenstellung auf Krocket bezieht, spiegelt sie grundlegende Prinzipien der autonomen Bewegungsplanung wider. In der Robotik müssen mobile Roboter zum Beispiel in der Fertigung, Logistik oder in Servicebereichen oft Wege berechnen, die mehrere Kontrollpunkte oder Hindernisse in einer festen Reihenfolge passieren. Ein Bezug dieser Aufgabe zur autonomen Bewegungsplanung bei Fußballspielenden Robotern kann ebenfalls hergestellt werden.

Während des Wettbewerbs entwarfen wir gemeinsam einen Brute-Force Algorithmus, also einen einfachen, aber aufgrund seiner exponentiellen Laufzeitkomplexität von $O(n^3)$, sehr ineffizienten Algorithmus, der alle Möglichkeiten überprüft bis eine korrekte Lösung gefunden wird. Dies ist ein Grund, warum mich diese Aufgabe auch nach dem Ende der ersten Runde des Wettbewerbs weiter gefesselt hat. Die Ineffizienz veranlasste mich dazu, auch nach dem Ende der ersten Runde allein weiter an einer effizienteren Lösung zu arbeiten. Die Vorarbeiten, welche Lennart und ich in der ersten Runde gemeinsam eingereicht hatten, habe ich im Rahmen meiner komplexen Leistung somit weiterentwickelt. Die damit einhergehende Optimierung stellte mich unter anderem vor diese Herausforderungen:
\begin{itemize}
	\item Die Identifikation und Behebung von Fehlern
	\item Das sorgfältige Abfangen von Randfällen (Edge Cases)
	\item Die Implementierung eines effizienteren Ansatzes, der den theoretischen Ansprüchen der Aufgabe gerecht wird.
\end{itemize}

Diese Herausforderungen sorgten nicht nur für eine spannende praktische Übung, sondern lieferten auch wertvolle Erkenntnisse für die theoretische Aufarbeitung. Darüber hinaus musste ich mir in Mathematik und Informatik einige Konzepte, welche nicht im Schulunterricht behandelt werden, selbst beibringen, da mir nicht alles im Vorfeld bekannt war. Dazu gehören zum Beispiel die Vektorrechnung und das Konzept der verzerrten Ebene (vgl. \ref{sec:ebene}). Bei Fragen und Problemen stand mir mein Wettbewerbsbetreuer Max Polter unterstützend zur Seite.

Die Zielstellung meiner Facharbeit besteht daher darin, den theoretischen Hintergrund der Aufgabe \enquote{Krocket} fundiert darzulegen und den von mir entwickelten linearen Lösungsansatz inklusive Implementierungsdetails und Beispieltests zu erläutern. Dabei sollen die wesentlichen Schwerpunkte der Arbeit – die detaillierte Analyse der Ausgangsproblematik und die Optimierung des Algorithmus – klar hervorgehoben werden.

Zusammenfassend bietet meine Facharbeit einen Einblick in eine Schnittstelle zwischen theoretischer Informatik und praktischer Anwendung – ein Aspekt, der mich auch nach der ersten Wettbewerbsrunde weiter in den Bann gezogen hat. Dabei soll meine Komplexe Leistung einen wissenschaftlich fundierten und zugleich praxisnahen Zugang zu einer komplexen Problemstellung bieten, in der ich einen linearen Lösungsansatz zur Optimierung des Problems entwickle.

